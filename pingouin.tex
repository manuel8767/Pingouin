\documentclass[12pt,a4paper]{article}
\usepackage[utf8]{inputenc}
\usepackage{lipsum}
\usepackage{geometry}
\usepackage{hyperref}
\geometry{margin=2.5cm}

\title{Manual de la Biblioteca \texttt{Pingouin} en Python}
\author{ }
\date{ }

\begin{document}

\maketitle

\tableofcontents
\newpage

\section{Introducción}
La librería \texttt{Pingouin} es un paquete estadístico para Python, diseñado para facilitar el análisis de datos de manera intuitiva, clara y con funciones de alto nivel. Proporciona herramientas para estadística descriptiva, pruebas de hipótesis, medidas de efecto, correlaciones, ANOVAs, regresión y análisis de confiabilidad. Su enfoque está en ser ligera, fácil de usar y producir resultados listos para ser interpretados.

\section{Características Generales}
\begin{itemize}
    \item Escrita en Python puro, sin dependencias pesadas.
    \item Produce resultados en formato de tabla (\texttt{DataFrame}) de pandas.
    \item Funciones de estadística avanzada integradas en pocas líneas de código.
    \item Compatible con librerías como \texttt{NumPy}, \texttt{SciPy} y \texttt{pandas}.
    \item Resultados listos para informes científicos (incluyendo valores p, intervalos de confianza y tamaños de efecto).
\end{itemize}

\section{Funciones Principales por Categoría}

\subsection{1. Estadísticos Descriptivos}
\begin{itemize}
    \item \texttt{pg.describe()}: genera estadísticas descriptivas de un conjunto de datos, como media, desviación estándar, error estándar, intervalo de confianza, asimetría y curtosis.
    \item \textbf{Ejemplo:}
    \begin{verbatim}
    pg.describe(data, dv='Nota', group='Grupo')
    \end{verbatim}
    \textbf{Resultados:} tamaño de la muestra, media, desviación típica, IC95\%, asimetría y curtosis por grupo.
\end{itemize}

\subsection{2. Pruebas de Normalidad y Homogeneidad}
\begin{itemize}
    \item \texttt{pg.normality()}: aplica la prueba de Shapiro-Wilk u otras pruebas para verificar si los datos siguen una distribución normal.
    \item \texttt{pg.homoscedasticity()}: realiza la prueba de Levene o Bartlett para comprobar la igualdad de varianzas entre grupos.
    \item \textbf{Ejemplo:}
    \begin{verbatim}
    pg.normality(data, dv='Nota', group='Grupo')
    pg.homoscedasticity(data, dv='Nota', group='Grupo')
    \end{verbatim}
\end{itemize}

\subsection{3. Pruebas de Hipótesis}
\begin{itemize}
    \item \texttt{pg.ttest()}: prueba t de Student para muestras independientes o relacionadas, con opción de corrección de Welch.
    \item \texttt{pg.mwu()}: prueba de Mann-Whitney U (alternativa no paramétrica a la t-test).
    \item \texttt{pg.anova()}: análisis de varianza de un factor.
    \item \texttt{pg.pairwise\_ttests()}: comparaciones múltiples con corrección por error tipo I.
    \item \textbf{Ejemplo:}
    \begin{verbatim}
    pg.ttest(dataA, dataB, correction='auto')
    \end{verbatim}
\end{itemize}

\subsection{4. Medidas de Efecto}
\begin{itemize}
    \item \texttt{pg.compute\_effsize()}: calcula el tamaño del efecto entre dos grupos (Cohen’s d, Hedges’ g, Glass’ delta, etc.).
    \item \texttt{pg.partial\_eta\_sq()}: calcula el eta cuadrado parcial para ANOVA.
    \item \textbf{Ejemplo:}
    \begin{verbatim}
    pg.compute_effsize(dataA, dataB, eftype='cohen')
    \end{verbatim}
\end{itemize}

\subsection{5. Correlaciones}
\begin{itemize}
    \item \texttt{pg.corr()}: calcula la correlación entre dos variables con diferentes métodos (Pearson, Spearman, Kendall).
    \item \texttt{pg.partial\_corr()}: correlación parcial controlando variables adicionales.
    \item \texttt{pg.rcorr()}: calcula una matriz completa de correlaciones.
    \item \textbf{Ejemplo:}
    \begin{verbatim}
    pg.corr(data['X'], data['Y'], method='pearson')
    \end{verbatim}
\end{itemize}

\subsection{6. ANOVA y Modelos Lineales}
\begin{itemize}
    \item \texttt{pg.anova()}: ANOVA de un factor o de medidas repetidas.
    \item \texttt{pg.rm\_anova()}: ANOVA de medidas repetidas.
    \item \texttt{pg.mixed\_anova()}: ANOVA de medidas mixtas (intra e inter sujetos).
    \item \textbf{Ejemplo:}
    \begin{verbatim}
    pg.anova(data=data, dv='Nota', between='Grupo')
    \end{verbatim}
\end{itemize}

\subsection{7. Fiabilidad y Consistencia}
\begin{itemize}
    \item \texttt{pg.cronbach\_alpha()}: calcula el alfa de Cronbach para medir consistencia interna.
    \item \texttt{pg.intraclass\_corr()}: calcula la correlación intraclase para fiabilidad entre evaluadores.
    \item \textbf{Ejemplo:}
    \begin{verbatim}
    pg.cronbach_alpha(data)
    \end{verbatim}
\end{itemize}

\subsection{8. Otras Funcionalidades Útiles}
\begin{itemize}
    \item \texttt{pg.rm\_corr()}: correlación de medidas repetidas.
    \item \texttt{pg.logistic\_regression()}: regresión logística.
    \item \texttt{pg.linear\_regression()}: regresión lineal simple o múltiple.
    \item \texttt{pg.power\_ttest()}: cálculo de potencia estadística en pruebas t.
    \item \texttt{pg.multicomp()}: corrección de comparaciones múltiples (Bonferroni, Holm, FDR, etc.).
\end{itemize}

\section{Conclusiones}
La librería \texttt{Pingouin} concentra en un solo paquete las funciones estadísticas más comunes para la investigación científica. Su diseño orientado a \texttt{pandas} y la facilidad de obtener resultados en tablas lo convierten en una herramienta ideal para psicología, neurociencia, biología, educación y áreas aplicadas que requieran análisis estadístico claro y reproducible.

\end{document}